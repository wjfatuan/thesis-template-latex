\chapter{Introducción}


%\addcontentsline{toc}{chapter}{Introduction}

La introducción consiste en una presentación general del problema de 
investigación que se abordó y su función es orientar al lector sobre 
el contenido general del documento que se presenta. En este apartado 
se incluyen algunos antecedentes relacionados en el trabajo, se hace 
explícito el problema abordado; se señalan los objetivos tanto generales 
como específicos y la manera cómo se desarrolló el trabajo para 
alcanzarlos (aspectos metodológicos), se incluyen algunos resultados 
obtenidos. Al finalizar se hace una descripción general del documento, 
capítulos y su contenido. No olvide citar sus fuentes usando el formato 
\gls{APA} \cite{style2020apa}, \cite{wiki:Plagiarism}, \cite{Grätzer2014}, y \cite{Goniwada2022}.
