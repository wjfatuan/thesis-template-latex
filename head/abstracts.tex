% Resumen en español
\begin{otherlanguage}{spanish}
    \chapter*{Resumen}
    \addcontentsline{toc}{chapter}{Resumen}

    El resumen presenta el contenido del documento de forma abreviada y precisa,
    no se hace en él ninguna interpretación. Recoja las ideas principales del texto
    y presente sus ideas de forma objetiva, clara, con las ideas fundamentales y
    aquellas secundarias que sean relevantes para el entendimiento de las
    primeras.

    Cuando redacte el resumen tenga en cuenta la estructura del texto original,
    céntrese en las ideas principales que reflejen la estructura del documento
    general, pero conserve las ideas centrales de este. Tenga en cuenta que dentro
    del resumen se redacta una pequeña introducción, luego un desarrollo y una
    conclusión de la exposición de sus ideas o conocimientos. En términos prácticos
    el resumen incluye tal como el texto original una introducción, la metodología
    del estudio (si es que aplica), el desarrollo, y una conclusión con
    recomendaciones (si es que estas aplican).

    Un buen resumen, además de reflejar en forma objetiva los contenidos del
    texto original, los explica en una forma más sencilla y utiliza conceptos de más
    fácil comprensión, ya que para ahondar en los temas, siempre se contará con
    el documento principal.

    \vskip0.5cm

    Palabras clave: palabra1, palabra2
\end{otherlanguage}

% English abstract
\begin{otherlanguage}{english}
    \chapter*{Abstract}
    \addcontentsline{toc}{chapter}{Abstract}
    % put your text here: 150 words max
    % put your text here
    \lipsum[1-2]
    \vskip0.5cm
    Keywords: keyword1, keyword2
\end{otherlanguage}
